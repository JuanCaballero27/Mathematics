\documentclass{tufte-handout}
\usepackage{amsmath,amssymb,amsthm}
\usepackage{mathtools}


\title{Course:  Real Analysis(Sequences)}
\author{Sebastián Caballero}
\date{22 Nov. 2022}

\usepackage{color-tufte}

\begin{document}
\maketitle

\begin{abstract}
\noindent
A simple notes template. Inspired by Tufte-\LaTeX class and beautiful notes by \begin{verbatim*}
	https://github.com/abrandenberger/course-notes
\end{verbatim*}
\end{abstract}

\section{Introduction to Sequences}
Generally, when someone starts to study mathematics and specially set theory, a sequence is defined as a function from $\mathbb{N}$ to a set $S$. In this case, we defined a sequence as a function to $\mathbb{R}$, and the properties of this sequences will help us in real Analysis further when we start our study on limits and infinite series.

\begin{definition}[Sequence of real numbers]
	Let $X$ be a function from $\mathbb{N}$ to $\mathbb{R}$ is called a \textbf{sequence of real numbers}, and notations are:
	\begin{itemize}
		\item For the sequence $X$, $f(n)$ is noted as $x_n$
		\item The sequence $X$ can be written as $(x_n)$, $(x_n: n \in \mathbb{N})$
	\end{itemize}
\end{definition}
The use of parentheses are for emphasize the order induced by the natural numbers, even when a function is just a set of ordered pairs and it gives the property that $\{(x, y), (a, b)\} = \{(a, b), (x, y)\}$. 

\end{document}