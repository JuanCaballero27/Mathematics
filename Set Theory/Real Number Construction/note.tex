\documentclass{tufte-handout}

\title{Course: Set Theory(Real Numbers)}
\author{Sebastián Caballero}
\date{December 4th 2022}

\usepackage{color-tufte}

\begin{document}
\maketitle

\begin{abstract}
\noindent
A simple notes template. Inspired by Tufte-\LaTeX class and beautiful notes by \begin{verbatim*}
	https://github.com/abrandenberger/course-notes
\end{verbatim*}
\end{abstract}
\section{The initial Problem}
What is a number? We have been constructed from a set focused way the natural, the integers and the rational numbers. They arise from relations and we gave them the usual structure we are comfortable with when we learn naively about them. And these results were already known in the 19th century, but a bigger problem arises when we try to define the real numbers.\\

In our intuition, the real numbers are just the rational numbers with the irrational numbers, and since we have already defined the rational numbers, we just need to think about the irrational numbers, like $\sqrt{2}, e, \pi$.\\

Many mathematicians tried to define what is a real number, among the most important are \textit{Cantor, Dedekind, and Cauchy}. We are going to explore their constructions and also prove that they are reaching the same structure in the end.

\section{Dedekind's Cuts}
\subsection{Defining a number}
Do you remember when you learnt about the real numbers? A good intuition for them is that they are in a line, and to each of the infinite points in the line we assign a real number in certain order. That is the intuition given for Dedekind to define real numbers, since when you think about a real number $r$, you put it in the real line and $r$ creates two disjoint sets:
\begin{align*}
	L &= \{x \in \mathbb{R} : x < r\}\\
	R &= \{x \in \mathbb{R} : x > r\}
\end{align*}
But wait, we can't use $\mathbb{R}$ to define $\mathbb{R}$, so Dedekind thought on define those sets just with $\mathbb{Q}$. We are going to focus on the $L$ set for real numbers, but it is easy to see that $R$ is just $\mathbb{Q} \setminus R$.\\

So, our need in the definition is that the set $L$ is determined only by rational numbers. It can't be empty and can't be all the rational numbers since we want that when we pick a real number, there are lower and upper numbers. Also, we want that the set has no maximum since we want to find between to real numbers, another real number. With that, let's announce our definition:
\begin{definition}[Dedekind's cut]
	A Dedekind's cut or just a cut, is a set $r$ such that:
	\begin{enumerate}
		\item $r \neq \emptyset$ and $r \subset \mathbb{Q}$.
		\item If $x \in r$ and $y \le_\mathbb{Q} x$ then $y \in r$. We say that $r$ is closed to left
		\item For all $x \in r$, there is an element $y$ such that $x <_\mathbb{Q} y$
	\end{enumerate}
	We call this set \textbf{a real number} and the set of all Dedekind's cut is called $\mathbb{R}$.
\end{definition}

Ok, in this way we have defined what can be a real number. Think for example that $\sqrt{2} = \{x \in \mathbb{Q}: x^2 <_\mathbb{Q} 2 \vee x <_\mathbb{Q} 0\}$. It is easy to check that this is a real number with our definition and we have just explained our first rational! But remember, we can't exclude the rational numbers and since the Dedekind's cuts are sets of rational, we need to explain what we call a rational number in the real numbers.
\begin{definition}[$\mathbb{Q}$ in $\mathbb{R}$]
	For any $q \in \mathbb{Q}$, we define the real number $q \in \mathbb{R}$ as:
	\begin{align*}
		q := \{x \in \mathbb{Q}: x <_\mathbb{Q} q\}
	\end{align*}
\end{definition}
So far, we have just defined our objects, and we want to explore later the properties of real numbers that are so familiar in courses like analysis.

\begin{problem}
	Is our definition of rational number in $\mathbb{R}$ a real number?
\end{problem}
\begin{proof}
	Indeed, $q$ is not an empty set since $q-\mathbb{Q} 1 <_\mathbb{Q} q$ and it is not $\mathbb{Q}$ since $q+_\mathbb{Q} 1 >_\mathbb{Q} q$. So it satisfies the first property. Thanks to the transitive property of $<_\mathbb{Q}$ we also have property two. And if you have $p \in q$, then define $x = \frac{p+_\mathbb{Q} q}{2}$ and note that $p < x < q$ so $x \in q$ and $p < x$. Therefore, $q$ is a Dedekind's cut.
\end{proof}

Before moving on, we are going just to prove that our real number set has two properties that show us how our definition is related to our intuition of the real line.
\begin{problem}
	Let $r$ be a Dedekind's cut.
	\begin{itemize}
		\item $r$ is bounded above as a subset of $\mathbb{Q}$, i.e. there is some $y \in \mathbb{Q}$ such that $x <_\mathbb{Q} y$ for all $x \in r$.
		\item $r$ and $\mathbb{Q} \setminus r$ forms a partition of $\mathbb{Q}$ and any element on $r$ is less than every element of $\mathbb{Q} \setminus r$.
	\end{itemize}
\end{problem}
\begin{proof}
	Suppose $r$ is a Dedekind's cut.
	\begin{itemize}
		\item Since $r \subset \mathbb{Q}$ there must be an $y \in \mathbb{Q}$ such that $y \not\in r$. Now, since $r$ is closed to left, if for any $x \in r$, $y \le_\mathbb{Q} x$ then $y \in r$ which is a contradiction. Thus $x <_\mathbb{Q} y$ for all $x \in r$
		\item It is evident that $r$ and $\mathbb{Q} \setminus r$ forms a partition of $\mathbb{Q}$ since they are disjoint, not empty and their union is $\mathbb{Q}$. Suppose $x \in r$ and $y \in \mathbb{Q} \setminus r$, if $y \le_\mathbb{Q} x$ then $y \in r$($r$ is closed to the left) but $y \in \mathbb{Q} \setminus r$ so it is a contradiction. Therefore $x <_\mathbb{Q} y$.
	\end{itemize}
\end{proof}

\subsection{An order for $\mathbb{R}$}
We have defined our real numbers as subsets of $\mathbb{Q}$. We want know to define an order for real numbers and equality. So, think that if $x < y$ and they are real numbers, that should implies that all rational number less than $x$ is also less than $y$. In our definition, it is just that $x \subseteq y$.
\begin{definition}[Order for $\mathbb{R}$]
	Let $x$ and $y$ be real numbers. We say that $x = y$ if they are equal as sets and we say that $x \le_\mathbb{R} y$ if $x \subseteq y$. When $x \le_\mathbb{R} y$ but $x \neq y$ we can write that $x <_\mathbb{R} y$.
\end{definition}

Ok, and just by complete our intuition about the numbers in a line, we must prove that $\le_\mathbb{R}$ is a linear order. That is:
\begin{itemize}
	\item $x \le_\mathbb{R} x$ for all $x \in \mathbb{R}$
	\item $x \le_\mathbb{R} y$ and $y \le_\mathbb{R} x$ implies that $x = y$
	\item $x \le_\mathbb{R} y$ and $y \le_\mathbb{R} z$ implies that $x \le_\mathbb{R} z$
	\item If $x \neq y$ then $x <_\mathbb{R} y$ or $y <_\mathbb{R} x$.
\end{itemize}
\begin{theorem}[Lineality of $\mathbb{R}$]
	The order defined with $\le_\mathbb{R}$ is linear
\end{theorem}
\begin{proof}
	Let $x, y, z$ be real numbers. Then:
	\begin{itemize}
		\item $x \le_\mathbb{R} x$ since $x \subseteq x$
		\item If $x \le_\mathbb{R} y$ and $y \le_\mathbb{R} x$ then $x \subseteq y$ but $y \subseteq x$, thus $x = y$
		\item If $x \le_\mathbb{R} y$ and $y \le_\mathbb{R} z$ then $x \subseteq y$ and $y \subseteq z$, so if $a \in x$ we conclude that $a \in y$, hence $a \in z$ and we conclude that $x \subseteq z$, so $x \le_\mathbb{R} z$
		\item Suppose that $x \neq y$ and $x \not\le_{\mathbb{R}} y$. That implies that exists $a \in x$ such that $a \not \in y$. So $a \in \mathbb{Q} \setminus y$ and for any $b \in y$, $b \le_\mathbb{Q} a$, and since $a \in x$ and $x$ is closed to the left, we can conclude that $b \in x$. So $y \subseteq x$ and $y \le_\mathbb{R} x$.
	\end{itemize}
\end{proof}

And now, thanks to the order in $\mathbb{R}$ we can talk about what makes real numbers really special. If you take any nonempty subset of $\mathbb{R}$ and it has a upper bound, then it has a supreme(Least upper bound). So, let's enunciate the definition first:
\begin{definition}[Supreme and infimum]
	Let $X$ be a set with an order relation $\le$.
	\begin{itemize}
		\item $u$ is an upper bound of $X$ if $x \le u$ for all $x \in X$
		\item $l$ is a lower bound of $X$ if $l \le x$ for all $x \in X$
		\item $s$ is the supreme of $X$ if it is an upper bound and if $u$ is also an upper bound then $s \le u$
		\item $i$ is the infimum of $X$ if it is a lower bound and if $l$ is also an upper bound then $l \le i$
	\end{itemize}
\end{definition}

With this definition, we can state \textit{the completess property} that is very useful in analysis courses.
\begin{theorem}[Completess of $\mathbb{R}$]
	If $A$ is a nonempty subset of $\mathbb{R}$ bounded above, then $A$ has a supreme.
\end{theorem}
\begin{proof}
	Suppose $A$ is a nonempty subset of $\mathbb{R}$ bounded above. Now consider $\bigcup A$ and we want to prove that this is the supreme of $A$. For that, we need to prove that $\bigcup A$ is a real number:
	\begin{itemize}
		\item Since $A$ is bounded above, there is $r \in \mathbb{R}$ such that $x \le r$ for all $x \in A$. So if $p \in A$, $p \in r$. Now, that means that for exists $q \in \mathbb{Q} \setminus r$. If $q \in \bigcup A$ then $q \in x$ for some $x\in A$, but then $q \in r$ and it implies that $q \not \in \mathbb{Q} \setminus r$ which is a contradiction, so $q \not \in \bigcup A$ and $\bigcup A \neq \mathbb{Q}$.
		\item If you take $q \in \bigcup A$ and $p \le_\mathbb{Q} q$ then there is $x \in A$ such that $q \in x$ and since $x$ is a real number, it is closed to the left so $p \in x$ and therefore $p \in \bigcup A$
		\item If you take $p \in \bigcup A$ then $p \in x$ for some $x \in A$. And since $x \in \mathbb{R}$ it has no maximum element, so there is a $q \in x$ such that $p \le_\mathbb{Q} q$, so $q \in \bigcup A$ and therefore $\bigcup A$ has no maximum element. 
	\end{itemize}
	Ok, it is a real number. Now, if you take any $x \in A$, $x \subseteq \bigcup A$ so $\bigcup A$ is an upper bound for $A$. Suppose $s$ is also an upper bound for $A$, so if $p \in \bigcup A$ then $p \in x$ for some $x \in A$ and since $x \le_\mathbb{R} s$ then $p \in s$. But it implies that $\bigcup A \le_\mathbb{R} s$ so $\bigcup A$ is the supreme of $A$. 
\end{proof}

Ok, we got so far, but there are sets in $\mathbb{R}$ such that they are not bounded. Indeed, $\mathbb{R}$ is not bounded and that is our next theorem:
\begin{theorem}
	$\mathbb{R}$ is not a bounded set
\end{theorem}
\begin{proof}
	Suppose in really $\mathbb{R}$ has an upper bound. Then there is a real number $s$ such that $r \le_{\mathbb{R}} s$ for all real number $r$. Now, since $s$ is a real number and it is nonempty subset of $\mathbb{Q}$ then there is a rational $q$ such that $q \in \mathbb{Q} \setminus s$. Now, we have the corresponding real number $q$ defined as:
	\begin{align*}
		q &:= \{x \in \mathbb{Q}: x \le_\mathbb{Q} q\}\\
	\end{align*}
	Since $q \in \mathbb{Q} \setminus s$ then $y \le_\mathbb{Q} q$ for all $y \in s$. That means that $y <_\mathbb{Q} q+1$ for all $y \in s$ and that implies that if we define $p = q+1$ then $s <_\mathbb{R} p$ so $s$ is not an upper bound.
\end{proof}

Ok, we are interested now in operations. So, the addition is just defined as:
\begin{definition}[Addition of real numbers]
	If $x, y \in \mathbb{Q}$ then $x +_\mathbb{R} y$ is defined as:
	\begin{align*}
		x +_\mathbb{R} y := \{p +_\mathbb{Q} q \in \mathbb{Q}: p \in x \wedge q \in y \}
	\end{align*}
\end{definition}

Ok, this is a simple definition for addition. Now, we need some tools for define the product of these numbers. But this isn't as simple as the addition. For example, if we want that $2 \cdot_\mathbb{R} 3 = 6$ then we would need that $(-1) \cdot_\mathbb{R} (-10) = 10$ be into the set defined for $6$. So, before getting into that, it is worth to look the properties of $+_\mathbb{R}$.
\begin{theorem}
	The set $\mathbb{R}$ with $+_\mathbb{R}$ is an abelian group.
\end{theorem} 
\begin{proof}
	For that, take $x, y, z \in \mathbb{R}$. 
	\begin{itemize}
		\item \textbf{Closure:} If you take $x$ and $y$ real numbers, the set $x +_\mathbb{R} y$ is not empty since $x$ and $y$ are not empty. Thanks to the monotony of $\mathbb{Q}$, it doesn't contains all rational numbers. Thanks again to the monotony of $\mathbb{Q}$ it is closed to the left and it has no maximum element since $x$ and $y$ don't have it. So $x +_\mathbb{R} y$ is also a real number.
		\item \textbf{Associativity:} Suppose that $a \in (x +_\mathbb{R} y) +_\mathbb{R} z$, then $a = p +_\mathbb{Q} q$ for some $p \in (x +_\mathbb{R} y)$ and $q \in z$. But then $p = r +_\mathbb{Q} s$ for some $r \in x$ and $s \in z$. And therefore $a =r +_\mathbb{Q} s +_\mathbb{Q} q$. And given the associativity in the rational numbers, we can express $a = r +_\mathbb{Q} (s +_\mathbb{Q} q)$, so $s +_\mathbb{Q} q \in y+_\mathbb{R} z$ and hence $a \in x +_\mathbb{R} (y +_\mathbb{R} z)$. You can use the same argument in the opposite direction and conclude that $x +_\mathbb{R} (y +_\mathbb{R} z) = (x +_\mathbb{R} y) +_\mathbb{R} z$.
		\item \textbf{Identity element}: For any $x \in \mathbb{R}$, the set $x +_\mathbb{R} 0$ is defined as:
		\begin{align*}
			x +_\mathbb{R} 0 &:= \{p + q: p \in x \wedge q \in 0\}
		\end{align*}
		Now, if $a \in x+_\mathbb{R} 0$ then $a = p + q$ with $p \in x$ and $q \in 0$. Since $q <_\mathbb{Q} 0$ then $p + q <_\mathbb{Q} p$ and therefore $a \in x$. Now, if $a \in x$ it is obvious that $a \in x+_\mathbb{R} 0$, so $x +_\mathbb{R} 0 = x$.
		\item \textbf{Inverse element:} For any $x \in \mathbb{R}$, we want to define $-x$ such that $x +_\mathbb{R} (-x) = 0$. Construct $-x$ as follows:
		\begin{align*}
			-x &:= \{p \in \mathbb{Q}: (\forall q \in x)(q <_\mathbb{Q} -p)\}
		\end{align*}
		This construction arises when you think in the Dedekind cuts on the real line and try to reflect it. First, we need to prove that this is indeed a real number. First, it is not empty since $\mathbb{Q} \setminus x$ has many elements greater than all element in $x$, so the inverse of that elements are in $-x$ and since $\mathbb{Q} \setminus x \neq \mathbb{Q}$ then $-x \neq \mathbb{Q}$. It is closed to the left, since if $r < s$ and $s \in -x$ then for all $q \in x$, $q <_\mathbb{Q} -s$ and since $-s < -r$ then $q <_\mathbb{Q} -r$ so $r \in -x$. Also, it has no maximum, since for $p \in -x$, $q <_\mathbb{Q} -p$ for all $q \in x$, and since $x$ is a real number it is impossible that $\mathbb{Q} \setminus x$ contains an infimum. Therefore there is $-s \in \mathbb{Q} \setminus x$ such that $-s <_\mathbb{Q} -p$ and $q <_\mathbb{Q} -s$ for all $q \in x$, so we have:
		\begin{align*}
			-s <_\mathbb{Q} \frac{-s-p}{2} <_\mathbb{Q} -p
		\end{align*}
		And since $-s <_\mathbb{Q} \frac{-s-p}{2}$ then $\frac{s+p}{2} \in -x$ and $p <_\mathbb{Q} \frac{s+p}{2}$. We just need to prove that $x + (-x) = 0$. If $a \in x +_\mathbb{R} (-x)$ then $a = p +_\mathbb{Q} q$ with $p \in x$ and $q \in -x$. Since $q \in -x$ then $p <_\mathbb{Q} -q$ so $p+_\mathbb{Q} q <_\mathbb{Q} 0$ and therefore $a \in 0$. If $a \in 0$, it is obvious that $a \in x +_\mathbb{R} (-x)$. So $x +_\mathbb{R} (-x) = 0$.
		\item \textbf{Commutativity:} For any $x, y \in \mathbb{R}$, we know that
		\begin{align*}
			x +_\mathbb{R} y &:= \{p +_\mathbb{Q} q : p \in x \wedge q \in y\} & y +_\mathbb{R} x &:= \{q +_\mathbb{Q} p: q \in y \wedge p \in x\}
		\end{align*}
		And at this point thanks to the commutativity of $+_\mathbb{Q}$ it should be evident that $x +_\mathbb{R} y = y +_\mathbb{R} x$. 
	\end{itemize}
\end{proof}

Good! So one of the most important results of the last theorem, is the existence of $-x$. So, we will define now the absolute value of a real number.
\begin{definition}[Absolute value]
	Let $x$ be a real number, then $|x|$ is defined as:
	\begin{align*}
		|x| &=: \begin{cases}
			x, & x \ge_\mathbb{R} 0\\
			-x, & x <_\mathbb{R} 0
		\end{cases}
	\end{align*}
	Taking $-x$ as the inverse of $x$ under $+_\mathbb{R}$.
\end{definition}

We know that the absolute value is a real number. But we want to ensure that $|x| \ge 0$ for any real number $x$. So, we will prove an important theorem:
\begin{theorem}[Monotony of $<_\mathbb{R}$ and $+_\mathbb{R}$]
	Let $x, y$ be real numbers such that $x \le_\mathbb{R} y$, then
	$$ x +_\mathbb{R} z \le_\mathbb{R} y +_\mathbb{R} z $$ for any $z \in \mathbb{R}$
\end{theorem}
\begin{proof}
	Since $x \le_\mathbb{R} y$, $x \subseteq y$. Let $z$ be any real number, the addition is defined as:
	\begin{align*}
		x +_\mathbb{R} z &:= \{p + q: p \in x \wedge q \in z\}
	\end{align*}
	So if $p + q \in x+_\mathbb{R} z$, we know that $p \in x$, but also $p \in y$. And therefore, $p + q \in y_\mathbb{R} z$ so we conclude that $x +_\mathbb{R} z \subseteq y +_\mathbb{R} z$.
\end{proof}

\begin{corollary}
	If $x$ is a real number, then $|x| \ge_\mathbb{R} 0$.
\end{corollary}
\begin{proof}
	If $x \ge_\mathbb{R} 0$ it is evident that. If $x \le_\mathbb{R} 0$ then
	\begin{align*}
		x &\le_\mathbb{R} 0\\
		x + (-x) &\le_\mathbb{R} 0 + (-x)\\
		0 &\le_\mathbb{R} -x\\
		0 &\le_\mathbb{R} |x|
	\end{align*}
	So the proposition is always true.
\end{proof}

\begin{definition}[Product of real numbers]
	Let $x, y$ be real numbers such that $x, y \ge_\mathbb{R} 0$, then $x \cdot_\mathbb{R} y$ is defined as:
	\begin{align*}
		x \cdot_\mathbb{R} y &:= \{p \cdot_\mathbb{Q} q: p \in x \wedge q \in y \wedge p, q \ge_\mathbb{Q} 0\} \cup 0
	\end{align*}
	And we define the product in any other case as:
	\begin{itemize}
		\item $-(|x| \cdot_{\mathbb{R}} y)$ if $x <_\mathbb{R} 0$ and $y \ge_\mathbb{R} 0$
		\item $-(x \cdot_\mathbb{R} |y|)$ if $x \ge_\mathbb{R} 0$ and $y <_\mathbb{R} 0$
		\item $|x| \cdot_\mathbb{R} |y|$ if $x, y <_\mathbb{R} 0$
	\end{itemize}
\end{definition}

With this definition, the first question is wether $x \cdot_\mathbb{R} y$ is also a real number. If any of them is $0$, then the product is $0$(Why?) and it is a real number. So, if they are not $0$, we know that thanks to the monotony of $\mathbb{Q}$ and the properties of $\mathbb{Q} \setminus x$ and $\mathbb{Q} \setminus y$, the set is not empty but also is not $\mathbb{Q}$. If $p \le_\mathbb{Q} q$ and $q \in x \cdot_\mathbb{R} y$, then $q = r \cdot_\mathbb{Q} s$. If $q \le_\mathbb{Q} 0$ then it is obvious that $p \le_\mathbb{Q} 0$ so in case it is not, $r \in x$, $s \in y$ and both are non-negative. So $s = \frac{q}{r}$. Now, $\frac{p}{r} \le_\mathbb{Q} \frac{q}{r} = s$, hence $\frac{p}{r} \in x$ and $\frac{p}{s} \cdot_\mathbb{Q} s = p \in x \cdot_\mathbb{R} y$. At last, for any $p \in x \cdot_\mathbb{R} y$, suppose that $p = r \cdot_\mathbb{Q} s$ with $r \in x$ and $s \in y$ and both are non-negative. Now, there are $t$ and $u$ such that $t \in x$, $u \in y$ and $r <_\mathbb{Q} t$, $s <_\mathbb{Q} u$, so for the monotony of $\mathbb{Q}$, $r \cdot_\mathbb{Q} s <_\mathbb{Q} t \cdot_\mathbb{Q} u$ and since $t \cdot_\mathbb{Q} u \in x \cdot_\mathbb{R}$ it is evident that this set has no maximum.\\

Just like the addition, the product form an abelian group under $\mathbb{R}$.
\begin{theorem}
	$\mathbb{R} \setminus \{0\}$ forms an abelian group under $\cdot_\mathbb{R}$
\end{theorem}
\begin{proof}
	If we prove that the positive real numbers form a group, the proof for the other cases is just use the properties of a group. 
	\begin{itemize}
		\item \textbf{Closure:} We have prove it in the comment before the theorem.
		\item \textbf{Associativity:} If $a \in x \cdot_\mathbb{R} (y \cdot_\mathbb{R} z)$ then $x \in 0$ or $x = p \cdot_\mathbb{Q} q$ for some $p \in x$ and $q \in y \cdot_\mathbb{R} z$. The first case is trivial, so suppose the second one. Since $q \in y \cdot_\mathbb{R} z$, $q \in 0$ or $q = r \cdot_\mathbb{Q} s$, for $r \in y$ and $s \in z$. If the second case is true, then $a = p \cdot_\mathbb{Q} r \cdot_\mathbb{Q} s$ and since $p \cdot_\mathbb{Q} r \in x \cdot_\mathbb{R} y$, $a \in (x \cdot_\mathbb{R} y) \cdot_\mathbb{R} z$. But if $q \in 0$, then $q <_\mathbb{Q} 0$ and hence $p \cdot q <_\mathbb{Q} 0$ so $x \in 0$ again. We conclude that $x \cdot_\mathbb{R} (y \cdot_\mathbb{R} z) \subseteq (x \cdot_\mathbb{R} y) \cdot_\mathbb{R} z$, and in a similar way you can prove the other way. So, $x \cdot_\mathbb{R} (y \cdot_\mathbb{R} z) = (x \cdot_\mathbb{R} y) \cdot_\mathbb{R} z$.
		\item \textbf{Identity element:} Let $x$ be any real number, our naive work says us that $1$ is the Identity element for $\cdot_\mathbb{R}$, so we will prove that it is true. Suppose $a \in x \cdot_\mathbb{R} 1$, and suppose that $a \not \in 0$, so $a = p \cdot_\mathbb{Q} q$ with $p \in x$ and $q \in 1$. Since $0 <_\mathbb{Q} q <_\mathbb{Q} 1$ then $0 <_\mathbb{Q} a <_\mathbb{Q} p$ and since $x$ is closed to the left, $a \in x$. Suppose $a \in x$, so if $a <_\mathbb{Q} 0$ it is obvious that $a \in x \cdot_\mathbb{R} 1$. If $a = 0$, then also is easy to see that $a \in x \cdot_\mathbb{R}1$, so suppose that $a >_\mathbb{Q} 0$.  
	\end{itemize}
\end{proof}

And with this, we just need a last property for real numbers that let us derive any other more. This is the distributive property.

\begin{theorem}[Ordered complete field]
	The set $\mathbb{R}$ with $\le_\mathbb{R}$, $+_\mathbb{R}$ and $\cdot_\mathbb{R}$ is an ordered complete field.
\end{theorem}

\end{document}