\documentclass{tufte-handout}

\title{Course: Set Theory(Integers \& Rationals)}
\author{Sebastián Caballero}
\date{\today}

\usepackage{color-tufte}

\begin{document}
\maketitle

\begin{abstract}
\noindent
A simple notes template. Inspired by Tufte-\LaTeX class and beautiful notes by \begin{verbatim*}
	https://github.com/abrandenberger/course-notes
\end{verbatim*}
\end{abstract}
\section{The need for new constructions}
Since we have constructed the natural numbers, we have put a lot of mathematics in hands of the set theory. But they are not sufficient in most cases, so we need new types of numbers. The first ones are the integers, in which we include the negative numbers and later we construct the rational, which are all the possible fractions between integers.

\section{What are integer numbers?}
\subsection{Origin of integers}
In the first approaches a kid have in mathematics, he will learn that $2-3$ is not possible since you can subtract more than what you have. Later, we are taught that there is a solution called $-1$, but this solution is also applicable to $3-4$ and so on. So, we have:
\begin{align*}
	3 - 2 &= 4 - 5
\end{align*}
And we can do this with a lot of natural numbers. But we don't have $-$ operation on natural number, so we are going to do it convenience. For two integers $a-b$ and $c-d$:
\begin{align*}
	a - b &= c - d\\
	a + d &= c + b
\end{align*}

And this will be our strategy.

\begin{definition}
	We define the relation $\sim$ over the set $\mathbb{N} \times \mathbb{N}$ such that $(a, b) \sim (c, d)$ if and only if $a + d = b + c$.
\end{definition}

This relation will associate all numbers whose difference is the same. For example, $(2, 3) \sim (4, 5)$ since $2 + 5 = 4 + 3$ which implies that $2 - 3 = 4 - 5$. 

\begin{theorem}
	The relation $\sim$ is an equivalence relation
\end{theorem}
\begin{proof}
	\begin{itemize}
		\item First, note that $a + b = a + b$ which by definition implies that $(a, b) \sim (a, b)$
		\item Suppose that $(a, b) \sim (c, d)$ so $a + d = c + b$ and then:
		\begin{align*}
			a + d &= c + b\\
			c + b &= a + d\\ 
		\end{align*}
		Therefore $(c, d) \sim (a, b)$
		\item Suppose that $(a, b) \sim (c, d)$ and $(c, d) \sim (x, y)$. This implies:
		\begin{align*}
			a + d &= c + b & c + y &= x + d\\
		\end{align*}
		If we add the two equalities we would have:
		\begin{align*}
			a + d +c + y &= c + b + x + d\\
			a + y + (c + d) &= x + b + (c + d)\\
			a + y &= x + b
		\end{align*}
		So we can conclude that $(a, b) \sim (x, y)$
	\end{itemize}
\end{proof}

We are done with our generalization! We are going to define just what is an integer.

\begin{definition}[Integers numbers]
	The set $(\mathbb{N} \times \mathbb{N}) / \sim$ will be called as the set of integer numbers and will be denoted as $\mathbb{Z}$. Any element in $\mathbb{Z}$ will be called an integer number.
\end{definition}

Our next tasks is to define the operations over integers. Take for example the integers $a-b$ and $c-d$ we can sum them as:
\begin{align*}
	a - b + c - d &= (a + c) - (b + d)
\end{align*}
So if $(a, b), (c, d) \in \mathbb{N} \times \mathbb{N}$ we define $(a, b) \circledcirc (c, d) = (a + c, b + d)$. Also, if we want to multiply numbers like those, we would do:
\begin{align*}
	(a-b)(c-d) &= ac - bc -ad + bd\\
	&= ac + bd - bc - ad\\
	&= (ac+bd) -(bc + ad)
\end{align*}
So we define $(a, b) \odot (c, d) = (ac+bd, bc+ad)$. We need then a theorem to show that this election is independent of what we choose in the equivalence classes.

\begin{theorem}
	Suppose that $(a, b) \sim (x, y)$ and $(c, d) \sim (z, w)$, then:
	\begin{itemize}
		\item $(a, b) \circledcirc (c, d) \sim (x, y) \circledcirc (z, w)$
		\item $(a, b) \odot (c, d) \sim (x, y) \odot (z, w)$
	\end{itemize}
\end{theorem}
\begin{proof}
	Assume that $(a, b) \sim (x, y)$ and $(c, d) \sim (z, w)$, we would have:
	\begin{align*}
		a + y &= b + x & c + w &= z + d
	\end{align*}
	And we have that:
	\begin{align*}
		(a, b) \circledcirc (c, d) &= (a + c, b + d)\\
		(x, y) \circledcirc (z, w) &= (x + z, y + w)
	\end{align*}
	And adding the equalities we have:
	\begin{align*}
		a + y + c + w &= b + x + z + d\\
		(a + c) + (y + w) &= (b + d) + (x + z)
	\end{align*}
	So we conclude that $(a + c, b + d) \sim (x + z, y + w)$. And for the $\odot$:
	\begin{align*}
		(a, b) \odot (c, d) &= (ac+bd, bc+ad)\\
		(x, y) \odot (c, d) &= (cx+dy, cy + dx)
	\end{align*}
	If we take the first equality from above we can manipulate as:
	\begin{align*}
		a + y &= b + x & a + y &= b + x\\
		ac + cy &= bc + cx & ad + dy &= bd + dx\\
		ac + cy &= bc + cx & bd + dx &= ad + dy
	\end{align*}
	And if we sum them up we would have:
	\begin{align*}
		ac + cy + bd + dx &= bc + cx + ad + dy\\
		(ac+bd) + (cy+dx) &= (cx+dy) + (bc + ad)
	\end{align*}
	And we conclude that $(a, b) \odot (c, d) \sim (x, y) \odot (c, d)$. In a similar way we can conclude that $(x, y) \odot (c, d) \sim (x, y) \odot (z, w)$ and therefore $(a, b) \odot (c, d) \sim (x, y) \odot (z, w)$.
\end{proof}

And with that we can define these operations in $\mathbb{Z}$ as:

\begin{definition}[Operations in $\mathbb{Z}$]
	Let $[(a, b)], [(c, d)] \in \mathbb{Z}$, we define:
	\begin{align*}
		[(a, b)] + [(c, d)] &= [(a, b) \circledcirc (c, d)]\\
		[(a, b)] \cdot [(c, d)] &= [(a, b) \odot (c, d)]
	\end{align*}
\end{definition}

Now, the important part of this definition is the structure for $\mathbb{Z}$.

\begin{theorem}[Structure of $\mathbb{Z}$]
	The set $\mathbb{Z}$ with $+$ and $\cdot$ forms a ring.
\end{theorem}
\begin{proof}
	\begin{itemize}
		\item The associative and commutative properties follows easily from the properties of $\mathbb{N}$. For a neutral element, we choose the class $[(0, 0)]$ and we would have that $[(a, b)] + [(0, 0)] = [(a, b)]$. For an inverse element of the integer $[(a, b)]$ we choose $[(b, a)]$ so that:
		\begin{align*}
			[(a, b)] + [(b, a)] &= [(a + b, a + b)]
		\end{align*}
		And since $(a+b, a+b) \sim (0, 0)$, it gives us the neutral element. So, we have an abelian group with $+$ under the set.
		\item The associative and commutative properties follows again from properties of $\mathbb{N}$, and we have an identity element $[(1, 0)]$ since we would have:
		\begin{align*}
			[(a, b)] \cdot [(1, 0)] &= [(a\cdot 1 + b\cdot 0, b\cdot 1 + a \cdot 0)]\\
			&= [(a + 0, b + 0)]\\
			&= [(a, b)]
		\end{align*}

		\item To prove that distributive law holds, take three items $[(a, b)], [(x, y)], [(z, w)]$.
		\begin{align*}
			[(a, b)] \cdot ([(x, y)] + [(z, w)]) &= [(a, b)] \cdot [(x+z, y+w)]\\
			&= [(ax+az+by+bw, ay+aw+bx+bz)]\\
			&= [(ax+by, ay+bx)] + [(az+bw, aw+bz)]\\
			&= [(a, b)] \cdot [(x, y)] + [(a, b)] \cdot [(z, w)]
		\end{align*}
	\end{itemize}
\end{proof}

The last property that $\mathbb{Z}$ misses for be a field is the existence of inverses in the operation $\cdot$. Far from being an annoying problem, this led to the development to new interesting mathematic. Now, what is also important to define is the order for integers. If we have that $a - b \le c - d$ then $a + d \le c + b$ which is what we are going to use to define the order.

\begin{definition}[Order in $\mathbb{Z}$]
	If $[(a, b)], [(c, d)] \in \mathbb{Z}$, we define $[(a, b)] \le [(c, d)]$ if and only if $a + d \le b + c$.
\end{definition}

This indeed is an order and is easy to see it with the properties of natural numbers. Reflexive and antisymmetric properties are obvious. For transitivity, suppose that $[(a, b)] \le [(c, d)]$ and $[(c, d)] \le [(x, y)]$, then:
\begin{align*}
	a + d &\le c + b & c + y \le x + d
\end{align*}
If we add the two inequalities:
\begin{align*}
	a + d + c + y &\le c + b + x + d\\
	a + y + (c + d) &\le x + b + (c + d)\\
	a + y &\le x + b
\end{align*}
So $[(a, b)] \le [(x, y)]$. The trichotomy property is just an use of the trichotomy in natural numbers. Now, we have the option to represent the integers just as natural numbers in a easy way, but this depends on a theorem.

\begin{theorem}
	Every integer number can be expressed as equivalence class with the form $[(n, 0)]$ or $[(0, n)]$ for $n \in \mathbb{N}$
\end{theorem}
\begin{proof}
	Suppose that $[(a, b)]$ is an integer number. Then there are three options:
	\begin{itemize}
		\item $a = b$, in this case $[(a, b)] = [(0, 0)]$
		\item $a < b$, so we have that there is $k \in \mathbb{N}$ such that $b = a + k$, and then $[(a, a+k)]$ which is the same as $[(0, k)]$ since $ a + k = a + k$. 
		\item $a > b$, so we have that there is $k \in \mathbb{N}$ such that $a = b + k$, and then $[(b+k, b)]$ is the same as $[(k, 0)]$ since $b + k = b + k$.
	\end{itemize}
\end{proof}

For this, we represent $0$ as $[(0, 0)]$, and we represent the integer $[(n, 0)]$ as $n$. Since $[(0, n)]$ would be the additive inverse we choose to represent it as $-m$. Note that the set $\mathbb{N}' := \{[(n, 0)] \in \mathbb{Z}: n \in \mathbb{N}\}$ is isomorphic to the set $\mathbb{N}$ so we say that $\mathbb{N}$ is a \textit{subset} of the integers, and we denote the sets $\mathbb{Z}* := \mathbb{Z} \setminus \{0\}$ and $\mathbb{Z}^+ := \mathbb{N}'$. Now, we can proof also some properties about order:
\begin{theorem}
	Let $x, y, z \in \mathbb{Z}$:
	\begin{itemize}
		\item If $x < y$ then $x + z < y + z$
		\item If $x < y$ and $z > 0$ then $xz < yz$
		\item If $x < y$ and $z < 0$ then $xz > yz$
		\item If $x \neq 0$, then $x \in \mathbb{Z}^+$ or $-x \in \mathbb{Z}^+$
		\item $x < y$ if and only if $y - x \in \mathbb{Z}^+$
	\end{itemize}
\end{theorem}
\begin{proof}
	Let $x, y, z \in \mathbb{Z}$, without loss of generality we can say that $x = [(a, b)]$, $y = [(c, d)]$ an $z = [(e, f)]$.
	\begin{itemize}
		\item If $x < y$ then $a + d < b + c$ so if we add $e, f$ to both sides, we would have $[(a + e, b+f)] < [(c+e, d+f)]$, which implies that $x + z < y + z$.
		\item If $x < y$ then $a + d < b + c$ and if $z > 0$, then $z = [(n, 0)]$ for some $n \in \mathbb{N}$. So, if we multiply by $n$ to both sides, we would have that $an+dn < bn+cn$. Also, we would have:
		\begin{align*}
			[(a, b)]\cdot [(n, 0)] &= [(a \cdot n + b \cdot 0, a \cdot 0 + b \cdot n)]\\
			&= [(a\cdot n, b\cdot n)]\\ 
			[(c, d)]\cdot [(n, 0)] &= [(c\cdot n + d\cdot 0, d \cdot n + c \cdot 0)]\\
			&= [(c\cdot n, d\cdot n)]
		\end{align*} 
		And therefore $x \cdot z < y \cdot z$.

		\item If $x < y$ then $a + d < b + c$ and if $z > 0$, then $z = [(0, n)]$ for some $n \in \mathbb{N}$. So, if we multiply by $n$ to both sides, we would have that $an + dn < bn + cn$. Also, we would have:
		\begin{align*}
			[(a, b)]\cdot [(0, n)] &= [(a \cdot 0 + b \cdot n, a \cdot n + b \cdot 0)]\\
			&= [(b\cdot n, a\cdot n)]\\ 
			[(c, d)]\cdot [(0, n)] &= [(c\cdot 0 + d\cdot n, d \cdot 0 + c \cdot n)]\\
			&= [(d\cdot n, c\cdot n)]
		\end{align*}
		And therefore $xz > yz$.

		\item This is a consequence of the definition of $\mathbb{Z}^+$ and the representation of integers.
		\item Suppose that $x < y$, this means that $a + d < b + c$. Note that $-x$ is $[(b, a)]$ and now we can sum it with $y$ so:
		\begin{align*}
			[(c, d)] + [(b, a)] &= [(c + b, d + a)]
		\end{align*}
		And if we compare to $[(0, 0)]$ since $a + d < b + c$ we would have that $y - x > 0$ and therefore it belongs to $\mathbb{N}^+$. The reverse implication is equivalent.
	\end{itemize}
\end{proof}

Integers have some more interesting properties! But they are not for our interest in this course, so we are going to continue with the next set we need.

\section{Rational Numbers}
In our everyday, we find expressions like \textit{This pencil has a lead of 0.5cm} or \textit{Give me half of the money}. These expressions are represented in mathematics in fractions, as a way to generalize the division for integers, and this structure is so important in algebra, number theory, and more; that is what we call rational numbers.\\

So, we think of them as fractions of the form $\frac{a}{b}$, like $\frac{1}{2}$. But also $\frac{1}{2}$ is the same as $\frac{2}{4}$ and $\frac{3}{6}$, so we can do a similar construction to which we did from natural numbers for integers. If two fractions $\frac{a}{b}$ and $\frac{c}{d}$ represent the same number, then $ad = bc$. We chose to construct this relation with all the coefficients except $0$ since we are going to construct a field over this structure.

\begin{definition}
	We define the relation $\equiv$ over $\mathbb{Z} \times \mathbb{Z}^*$ in a way such that $(a, b) \equiv (c, d)$ if and only if $a \cdot d = b \cdot c$.
\end{definition}

And what we have done is to relate all the elements which represent the same fraction.

\begin{theorem}
	$\equiv$ is an equivalence relation
\end{theorem}
\begin{proof}
	\begin{itemize}
		\item Note that $a \cdot b = a \cdot b$ so $(a, b) \equiv (a, b)$
		\item Suppose that $(a, b) \equiv (c, d)$, then $a \cdot d = b \cdot c$, so we can conclude that $c \cdot b = d \cdot a$ and therefore $(c, d) \equiv (a, b)$
		\item Suppose that $(a, b) \equiv (c, d)$ and $(c, d) \equiv (x, y)$. Then $a \cdot d = b \cdot c$ and $c \cdot y = d \cdot x$. If we multiply both sides we would have
		\begin{align*}
			a \cdot d \cdot c \cdot y &= b \cdot c \cdot d \cdot x\\
			a \cdot y \cdot (c \cdot d) &= b \cdot x \cdot (c \cdot d)\\
			a \cdot y &= b \cdot x
		\end{align*}
		And we conclude that $(a, b) \equiv (x, y)$.
	\end{itemize}
\end{proof}

Now, we have just a easy way to define what is a rational number in a similar way to what we did for integers with the relation we have constructed.

\begin{definition}[Rational numbers]
	The set $(\mathbb{Z} \times \mathbb{Z}^*)/\equiv$ will be denoted as $\mathbb{Q}$, the set of rational numbers and we call any element of $\mathbb{Q}$ as a rational number.
\end{definition}

And what we need to complete the construction is operations over $\mathbb{Q}$. From our intuition, if we have $\frac{a}{b}$ and $\frac{c}{d}$ then
\begin{align*}
	\frac{a}{b} + \frac{c}{d} &= \frac{ad+bc}{bd}
\end{align*}
And also
\begin{align*}
	\frac{a}{b} \cdot \frac{c}{d} &= \frac{ac}{bd}
\end{align*}
And since by our construction $\frac{a}{b}$ is $(a, b)$ then let us define the operations $\oplus$ and $\otimes$ as:
\begin{align*}
	(a, b) \oplus (c, d) &= (ad+bc, bd) & (a, b) \otimes (c, d) &= (ac, bd)
\end{align*}
If we prove that these operations are independent of the choice for $(a, b)$ and $(c, d)$ in the equivalence classes, we can take it as the definition for $+$ and $\cdot$ over $\mathbb{Q}$.

\begin{theorem}
	If $(a, b), (c, d), (x, y), (z, w) \in \mathbb{Z} \times \mathbb{Z}^*$ such that $(a, b) \equiv (x, y)$ and $(c, d) \equiv (z, w)$ then:
	\begin{itemize}
		\item $(a, b) \oplus (c, d) \equiv (x, y) \oplus (z, w)$
		\item $(a, b) \otimes (c, d) \equiv (x, y) \otimes (z, w)$
	\end{itemize}
\end{theorem}
\begin{proof}
	Note that the fact that $(a, b) \equiv (x, y)$ and $(c, d) \equiv (z, w)$ implies that:
	\begin{align*}
		ay &= bx & cw &= dz
	\end{align*}
	For the first two operations we have:
	\begin{align*}
		(a, b) \oplus (c, d) &= (ad + bc, bd)\\
		(x, y) \oplus (z, w) &= (xw + yz, yw)
	\end{align*}
	So, if we take the first equality we can operate like:
	\begin{align*}
		ay &= bx\\
		ayd^2 &= bxd^2\\
		ayd^2 + bcdy &= xbd^2 + bcdy
	\end{align*}
	And this implies that $(a, b) \oplus (c, d) \equiv (x, y) \oplus (c, d)$. In a similar way we can prove that $(a, b) \oplus (c, d) \equiv (a, b) \oplus (z, w)$ and by transitivity we would have that $(a, b) \oplus (c, d) \equiv (x, y) \oplus (z, w)$.\\

	Now, for the other two operations we would have:
	\begin{align*}
		(a, b) \otimes (c, d) &= (ac, bd)\\
		(x, y) \otimes (z, w) &= (xz, yw)
	\end{align*}

	And if we multiply the first two equalities as follows we would have:
	\begin{align*}
		ac \cdot yw &= bd \cdot xz\\
		acyw &= bdxz
	\end{align*}
	And we conclude that $(a, b) \otimes (c, d) \equiv (x, y) \otimes (z, w)$.
\end{proof}

And then we can define the operations over $\mathbb{Q}$:
\begin{definition}[Operations over $\mathbb{Q}$]
	Let $[(a, b)]$ and $[(c, d)]$ be rational numbers, we define:
	\begin{align*}
		[(a, b)] + [(c, d)] &= [(a, b) \oplus (c, d)]\\
		[(a, b)] \cdot [(c, d)] &= [(a, b) \otimes (c, d)]
	\end{align*}
\end{definition}

And this give us an important structure! A field!

\begin{theorem}
	The operations $+$ and $\cdot$ forms a field over $\mathbb{Q}$
\end{theorem}
\begin{proof}
	\begin{itemize}
		\item For rational numbers, the commutative and associative properties follows from the properties of $\mathbb{Z}$. The identity element $[(0, 1)]$ because:
		\begin{align*}
			[(a, b)] + [(0, 1)] &= [(a\cdot 1 + b \cdot 0, b \cdot 1)]\\
			&= [(a + 0, b)]\\
			&= [(a, b)]
		\end{align*}
		And for an inverse element for $[(a, b)]$ we have $[(-a, b)] = [(b, -a)]$(Because of properties for rings in $\mathbb{Z}$). This can be view since:
		\begin{align*}
			[(a, b)] + [(-a, b)] &= [(a \cdot b - a \cdot b, b \cdot b)]\\
			&= [(0, b^2)]\\
			&= [(0, 1)]
		\end{align*}

		\item The commutative and associative properties for $\cdot$ follows again from properties from $\mathbb{Z}$. The identity element for this operation is given by $[(1, 1)]$ because:
		\begin{align*}
			[(a, b)] \cdot [(1, 1)] &= [(a \cdot 1, b \cdot 1)]\\
			&= [(a, b)]
		\end{align*}
		And for the inverse element of $[(a, b)]$, $[(b, a)]$:
		\begin{align*}
			[(a, b)] \cdot [(b, a)] &= [(a\cdot b, b \cdot a)]\\
			&= [(a \cdot b, a \cdot b)]\\
			&= [(1, 1)]
		\end{align*}

		\item For the distributive property use:
		\begin{align*}
			[(a, b)] \cdot ([(x, y)] + [(z, w)]) &= [(a, b)] \cdot [(xw+yz, yw)]\\
			&= [(axw+ayz, byw)]\\
			&= [(abxw + abyz, b^2yw)]\\
			&= [(ax, by)] + [(az, bw)]\\
			&= [(a, b)] \cdot [(x,y)] + [(a, b)] \cdot [(z, w)]
		\end{align*}
		So this holds for all rational numbers.
	\end{itemize}
\end{proof}

And the last thing we need to define is an order for rational numbers. Think that a fraction is less than other, so $\frac{a}{b} \le \frac{c}{d}$ then $ad \le bc$. And that is what we are going to use.

\begin{definition}[Order in $\mathbb{Q}$]
	Suppose that $[(a, b)]$ and $[(c, d)]$ are rational numbers with $b, d \in \mathbb{Z}^+ \cup \{0\}$. We say that $[(a, b)] \le [(c, d)]$ if and only if $ad \le bc$.
\end{definition}

We can prove that this define an order:
\begin{theorem}
	$\le$ creates an order relation over $\mathbb{Q}$
\end{theorem}
\begin{proof}
	Reflexive and antisymmetric are obvious. For transitivity suppose that $[(a, b)] \le [(c, d)]$ and $[(c, d)] \le [(x, y)]$, then:
	\begin{align*}
		ad &\le bc & cy &\le dx
	\end{align*}
	And if we multiply by $y$ the first inequality and by $b$ the second one:
	\begin{align*}
		ady &\le bcy & bcy&\le bdx
	\end{align*}
	And by the transitivity of $\le$ in $\mathbb{Z}$ so:
	\begin{align*}
		ady &\le bdx\\
		ay &\le bx
	\end{align*}
	And therefore $[(a, b)] \le [(x, y)]$. And the linearlity of $\le$ in $\mathbb{Q}$ is a direct consequence of the linearlity of $\le$ in $\mathbb{Z}$.
\end{proof}

If we denote $0 = [(0, 0)]$ can define in a similar way as with the integers new sets with the order like $\mathbb{Q}^* = \mathbb{Q} \setminus \{0\}$ and also the set $\mathbb{Q}^+ = \{x \in \mathbb{Q}: x > 0\}$. Also, we can define an isomorphic set with $\mathbb{Z}$ as $\mathbb{Z}' = \{[(m, 1)] \in \mathbb{Q}: m \in \mathbb{Z}\}$, so we say that the integers are a \textit{subset} of the rational numbers. We end this chapter with the last properties of rational numbers and we introduce the notation $\frac{p}{q}$ for the $[(p, q)]$ in way that $[(a, b)] = [(x, y)]$ then $\frac{a}{b} = \frac{x}{y}$.

\begin{theorem}
	Suppose that $x, y, z \in \mathbb{Q}$ then:
	\begin{itemize}
		\item If $x \le y$ then $x + z \le y + z$
		\item If $x \le y$ and $z > 0$ then $x \cdot z \le y \cdot z$
		\item If $x \le y$ and $z < 0$ then $x \cdot z \ge y \cdot z$
		\item $x < y$ if and only if $y - x \in \mathbb{Q}^+$
		\item If $y > 0$ then there is $n \in \mathbb{N}$ such that $x < yn$
		\item If $x \le y$ then there exists $a \in \mathbb{Q}$ such that $x \le a \le y$
	\end{itemize}
\end{theorem}
\begin{proof}
	We can suppose that $x = \frac{a}{b}$, $y = \frac{c}{d}$ and $z = \frac{e}{f}$
	\begin{itemize}
		\item If $x \le y$ then $ad \le bc$ and we can manipulate it as:
		\begin{align*}
			ad &\le bc\\
			adf^2 &\le bcf^2\\
			adf^2 + bdfe &\le bcf^2 + bdfe\\
			df(af+be) &\le bf(cf+de)
		\end{align*}
		So we conclude that $\frac{af+be}{bf} \le \frac{cf+de}{df}$ so $x + z \le y + z$.

		\item If $\frac{e}{f} > 0$ then $e > 0$ and since $ad \le bc$ then $ade \le bce$ and since $f > 0$ by definition, $adef \le bcef$ so $x \cdot z \le y \cdot z$
		\item If $\frac{e}{f} < 0$ then $e < 0$ and since $ad \le bc$ then $ade > bce$ and since $f > 0$ by definition, $adef > bcef$ so $x \cdot z \ge y \cdot z$
		\item If $x < y$ and $ad < bc$ and if we add $0 < bc - ad$ and since $\frac{c}{d} - \frac{a}{b} = \frac{ad-bc}{bd}$ and by definition it is greater than $0$ so $y - x \in \mathbb{Q}^+$. The other side is the reverse implications.
		\item If $x \le 0$ then $n = 1$ would be enough. If not, then note that $\frac{a}{b} \le a$ since $a \le a\cdot b$ because $1 \le b$. And since $\frac{a}{b} > 0$ then $a \ge 1$ and therefore $a \le a \cdot c$. And note that $a \cdot c= a\cdot d \cdot y$. Putting altogether we would have:
		\begin{align*}
			\frac{a}{b} &\le a\\
			&\le a \cdot c\\
			&= ady\\
			&< ady + y\\
			&= (ad+1)y
		\end{align*}

		\item Take $a = \frac{1}{2} (x + y)$ which is:
		\begin{align*}
			a &= \frac{1}{2} \left(\frac{a}{b} + \frac{c}{d}\right)\\
			&= \frac{1}{2} \left(\frac{ad+bc}{bd}\right)\\
			&= \frac{ad+bc}{2bd}
		\end{align*}
		Note that due to the fact that $x \le y$ we have that:
		\begin{align*}
			ad &\le bc\\
			abd &\le b^2c\\
			2abd &\le abd + b^2c\\
			2b(ad) &\le b(ad + bc)
		\end{align*}
		And we can conclude that $\frac{a}{b} \le \frac{ad+bc}{2bd} = a$. In a similar way we can prove that $a \le y$.
	\end{itemize}
\end{proof}
\end{document}